% Metódy inžinierskej práce

\documentclass[10pt,twoside,slovak,a4paper]{article}

\usepackage[slovak]{babel}
%\usepackage[T1]{fontenc}
\usepackage[IL2]{fontenc} % lepšia sadzba písmena Ľ než v T1
\usepackage[utf8]{inputenc}
\usepackage{graphicx}
\usepackage{url} % príkaz \url na formátovanie URL
\usepackage{hyperref} % odkazy v texte budú aktívne (pri niektorých triedach dokumentov spôsobuje posun textu)
\usepackage[colorinlistoftodos,textsize=tiny]{todonotes}


\usepackage{cite}
%\usepackage{times}

\pagestyle{headings}

\title{Spotify recommendation system and its flaws\thanks{Semestrálny projekt v predmete Metódy inžinierskej práce, ak. rok 2024/25, vedenie: Mgr. Yevheniia Kataieva, PhD.}} % meno a priezvisko vyučujúceho na cvičeniach

\author{Timon Lumír Fillo\\[2pt]
	{\small Slovenská technická univerzita v Bratislave}\\
	{\small Fakulta informatiky a informačných technológií}\\
	{\small \texttt{xfillo@stuba.sk}}
	}

\date{\small \today} % upravte


\begin{document}

\maketitle

\begin{abstract}
ababbaabbaba
\end{abstract}



\section{Úvod}

Motivujte čitateľa a vysvetlite, o čom píšete. Úvod sa väčšinou nedelí na časti.

Uveďte explicitne štruktúru článku. Tu je nejaký príklad.
Základný problém, ktorý bol naznačený v úvode, je podrobnejšie vysvetlený v časti~\ref{recommendationsystem}.
Dôležité súvislosti sú uvedené v častiach~\ref{dolezita} a~\ref{dolezitejsia}.
Záverečné poznámky prináša časť~\ref{zaver}.


% Pointa je ukázať ako spotify recommendation system zlyháva pri skladateľoch s menším počtom stremov a naopak preferuje odporúčať skladateľov, ktorý sú vo svojich žánroch velikáni.
% Výsledkom tohto je že pre použivateľov, ktorý od tohto recommendation systemu očakávajú rozšírenie obzorov v oblasti alternatívnych žanrov alebo majú jednoducho napočúvané toľko že im spotify nevie odporučiť nič nové a cyklicky odporúča to isté
% Kritika ze recommendation system sa zakladá na preferencií ostatnyćh použivateľov a väčšina použivateľov počúva hudbu podľa trendov a nie podľa preferencie a tým pádom presnosť odporúčanie pre skupinu, ktorá sa trendom snaží vyhýbať
\section{Recommendation system} \label{recommendationsystem}
% Ako funguje, na akých dátach vyhodnocuje odporúčanie
% Prečo je vyhodnocovanie týchto dát mylné a ako to ovplyvňuje celkovy UX
\cite{hodgson2021spotify,}
Nejaky text
\todo{Blablbla}
a tak dalej
\section{Genre representation} \label{genre}
\subsection{Genre majorities} \label{genre:majorities}
\subsection{Genre minorities} \label{genre:minotiries}

\section{Cyclic recommentation} \label{cyclic}



\section{Záver} \label{zaver} % prípadne iný variant názvu
% Návrh na zlepšenie a zbavenie sa problémmov


%\acknowledgement{Ak niekomu chcete poďakovať\ldots}


% týmto sa generuje zoznam literatúry z obsahu súboru literatura.bib podľa toho, na čo sa v článku odkazujete
\newpage
\listoftodos[Notes]
\bibliographystyle{plain} % prípadne alpha, abbrv alebo hociktorý iný
\bibliography{literatura}

\end{document}
